\documentclass[]{article}
\usepackage{lmodern}
\usepackage{amssymb,amsmath}
\usepackage{ifxetex,ifluatex}
\usepackage{fixltx2e} % provides \textsubscript
\ifnum 0\ifxetex 1\fi\ifluatex 1\fi=0 % if pdftex
  \usepackage[T1]{fontenc}
  \usepackage[utf8]{inputenc}
\else % if luatex or xelatex
  \ifxetex
    \usepackage{mathspec}
  \else
    \usepackage{fontspec}
  \fi
  \defaultfontfeatures{Ligatures=TeX,Scale=MatchLowercase}
\fi
% use upquote if available, for straight quotes in verbatim environments
\IfFileExists{upquote.sty}{\usepackage{upquote}}{}
% use microtype if available
\IfFileExists{microtype.sty}{%
\usepackage{microtype}
\UseMicrotypeSet[protrusion]{basicmath} % disable protrusion for tt fonts
}{}
\usepackage[margin=1in]{geometry}
\usepackage{hyperref}
\hypersetup{unicode=true,
            pdftitle={R Notebook},
            pdfborder={0 0 0},
            breaklinks=true}
\urlstyle{same}  % don't use monospace font for urls
\usepackage{color}
\usepackage{fancyvrb}
\newcommand{\VerbBar}{|}
\newcommand{\VERB}{\Verb[commandchars=\\\{\}]}
\DefineVerbatimEnvironment{Highlighting}{Verbatim}{commandchars=\\\{\}}
% Add ',fontsize=\small' for more characters per line
\usepackage{framed}
\definecolor{shadecolor}{RGB}{248,248,248}
\newenvironment{Shaded}{\begin{snugshade}}{\end{snugshade}}
\newcommand{\AlertTok}[1]{\textcolor[rgb]{0.94,0.16,0.16}{#1}}
\newcommand{\AnnotationTok}[1]{\textcolor[rgb]{0.56,0.35,0.01}{\textbf{\textit{#1}}}}
\newcommand{\AttributeTok}[1]{\textcolor[rgb]{0.77,0.63,0.00}{#1}}
\newcommand{\BaseNTok}[1]{\textcolor[rgb]{0.00,0.00,0.81}{#1}}
\newcommand{\BuiltInTok}[1]{#1}
\newcommand{\CharTok}[1]{\textcolor[rgb]{0.31,0.60,0.02}{#1}}
\newcommand{\CommentTok}[1]{\textcolor[rgb]{0.56,0.35,0.01}{\textit{#1}}}
\newcommand{\CommentVarTok}[1]{\textcolor[rgb]{0.56,0.35,0.01}{\textbf{\textit{#1}}}}
\newcommand{\ConstantTok}[1]{\textcolor[rgb]{0.00,0.00,0.00}{#1}}
\newcommand{\ControlFlowTok}[1]{\textcolor[rgb]{0.13,0.29,0.53}{\textbf{#1}}}
\newcommand{\DataTypeTok}[1]{\textcolor[rgb]{0.13,0.29,0.53}{#1}}
\newcommand{\DecValTok}[1]{\textcolor[rgb]{0.00,0.00,0.81}{#1}}
\newcommand{\DocumentationTok}[1]{\textcolor[rgb]{0.56,0.35,0.01}{\textbf{\textit{#1}}}}
\newcommand{\ErrorTok}[1]{\textcolor[rgb]{0.64,0.00,0.00}{\textbf{#1}}}
\newcommand{\ExtensionTok}[1]{#1}
\newcommand{\FloatTok}[1]{\textcolor[rgb]{0.00,0.00,0.81}{#1}}
\newcommand{\FunctionTok}[1]{\textcolor[rgb]{0.00,0.00,0.00}{#1}}
\newcommand{\ImportTok}[1]{#1}
\newcommand{\InformationTok}[1]{\textcolor[rgb]{0.56,0.35,0.01}{\textbf{\textit{#1}}}}
\newcommand{\KeywordTok}[1]{\textcolor[rgb]{0.13,0.29,0.53}{\textbf{#1}}}
\newcommand{\NormalTok}[1]{#1}
\newcommand{\OperatorTok}[1]{\textcolor[rgb]{0.81,0.36,0.00}{\textbf{#1}}}
\newcommand{\OtherTok}[1]{\textcolor[rgb]{0.56,0.35,0.01}{#1}}
\newcommand{\PreprocessorTok}[1]{\textcolor[rgb]{0.56,0.35,0.01}{\textit{#1}}}
\newcommand{\RegionMarkerTok}[1]{#1}
\newcommand{\SpecialCharTok}[1]{\textcolor[rgb]{0.00,0.00,0.00}{#1}}
\newcommand{\SpecialStringTok}[1]{\textcolor[rgb]{0.31,0.60,0.02}{#1}}
\newcommand{\StringTok}[1]{\textcolor[rgb]{0.31,0.60,0.02}{#1}}
\newcommand{\VariableTok}[1]{\textcolor[rgb]{0.00,0.00,0.00}{#1}}
\newcommand{\VerbatimStringTok}[1]{\textcolor[rgb]{0.31,0.60,0.02}{#1}}
\newcommand{\WarningTok}[1]{\textcolor[rgb]{0.56,0.35,0.01}{\textbf{\textit{#1}}}}
\usepackage{graphicx,grffile}
\makeatletter
\def\maxwidth{\ifdim\Gin@nat@width>\linewidth\linewidth\else\Gin@nat@width\fi}
\def\maxheight{\ifdim\Gin@nat@height>\textheight\textheight\else\Gin@nat@height\fi}
\makeatother
% Scale images if necessary, so that they will not overflow the page
% margins by default, and it is still possible to overwrite the defaults
% using explicit options in \includegraphics[width, height, ...]{}
\setkeys{Gin}{width=\maxwidth,height=\maxheight,keepaspectratio}
\IfFileExists{parskip.sty}{%
\usepackage{parskip}
}{% else
\setlength{\parindent}{0pt}
\setlength{\parskip}{6pt plus 2pt minus 1pt}
}
\setlength{\emergencystretch}{3em}  % prevent overfull lines
\providecommand{\tightlist}{%
  \setlength{\itemsep}{0pt}\setlength{\parskip}{0pt}}
\setcounter{secnumdepth}{0}
% Redefines (sub)paragraphs to behave more like sections
\ifx\paragraph\undefined\else
\let\oldparagraph\paragraph
\renewcommand{\paragraph}[1]{\oldparagraph{#1}\mbox{}}
\fi
\ifx\subparagraph\undefined\else
\let\oldsubparagraph\subparagraph
\renewcommand{\subparagraph}[1]{\oldsubparagraph{#1}\mbox{}}
\fi

%%% Use protect on footnotes to avoid problems with footnotes in titles
\let\rmarkdownfootnote\footnote%
\def\footnote{\protect\rmarkdownfootnote}

%%% Change title format to be more compact
\usepackage{titling}

% Create subtitle command for use in maketitle
\newcommand{\subtitle}[1]{
  \posttitle{
    \begin{center}\large#1\end{center}
    }
}

\setlength{\droptitle}{-2em}

  \title{R Notebook}
    \pretitle{\vspace{\droptitle}\centering\huge}
  \posttitle{\par}
    \author{}
    \preauthor{}\postauthor{}
    \date{}
    \predate{}\postdate{}
  

\begin{document}
\maketitle

\begin{Shaded}
\begin{Highlighting}[]
\KeywordTok{library}\NormalTok{(Matrix); }\KeywordTok{library}\NormalTok{(MASS); }\KeywordTok{library}\NormalTok{(factoextra)}
\end{Highlighting}
\end{Shaded}

\begin{verbatim}
## Loading required package: ggplot2
\end{verbatim}

\begin{verbatim}
## Welcome! Related Books: `Practical Guide To Cluster Analysis in R` at https://goo.gl/13EFCZ
\end{verbatim}

\begin{Shaded}
\begin{Highlighting}[]
\KeywordTok{library}\NormalTok{(tidyverse); }\KeywordTok{library}\NormalTok{(cluster)}
\end{Highlighting}
\end{Shaded}

\begin{verbatim}
## -- Attaching packages --------------------------------------------------------------------------- tidyverse 1.2.1 --
\end{verbatim}

\begin{verbatim}
## v tibble  2.0.1     v purrr   0.2.5
## v tidyr   0.8.2     v dplyr   0.7.8
## v readr   1.3.1     v stringr 1.3.1
## v tibble  2.0.1     v forcats 0.3.0
\end{verbatim}

\begin{verbatim}
## -- Conflicts ------------------------------------------------------------------------------ tidyverse_conflicts() --
## x tidyr::expand() masks Matrix::expand()
## x dplyr::filter() masks stats::filter()
## x dplyr::lag()    masks stats::lag()
## x dplyr::select() masks MASS::select()
\end{verbatim}

\begin{Shaded}
\begin{Highlighting}[]
\NormalTok{zip_file_path =}\StringTok{ "./data/award_words.zip"}
\KeywordTok{unzip}\NormalTok{(zip_file_path)}

\NormalTok{words =}\StringTok{ }\KeywordTok{read.csv}\NormalTok{(}\StringTok{"words.csv"}\NormalTok{, }\DataTypeTok{stringsAsFactors =} \OtherTok{FALSE}\NormalTok{)}
\NormalTok{weights =}\StringTok{ }\KeywordTok{read.csv}\NormalTok{(}\StringTok{"weights.csv"}\NormalTok{, }\DataTypeTok{stringsAsFactors =} \OtherTok{FALSE}\NormalTok{)}
\NormalTok{agencies =}\StringTok{ }\KeywordTok{read.csv}\NormalTok{(}\StringTok{"agencies.csv"}\NormalTok{, }\DataTypeTok{stringsAsFactors =} \OtherTok{FALSE}\NormalTok{)}
\end{Highlighting}
\end{Shaded}

\textbf{1. What are the advantages and disadvantages of using lookup
tables to represent the agencies and words, compared to storing
everything in one table?}

\textbf{2. Compute the sizes of the following objects algebraically, and
verify the sizes in R. Recall from lecture that there may be around 1 KB
memory overhead per object, so your theoretical results will not match
exactly.}

Weight of triple representation

\[ n * s_i + n * s_i + n * s_d \] \[ 145834 * (4 + 4 + 8) = 23336544\]

Weight of sparse matrix \[ n * s_d + n * s_i + d * s_i \]
\[ 145834 * 8 + 145834 * 4 + 243 * 4 = 17504880\] Weight of sparse
matrix transpose \[ n * s_d + w * s_i + n * s_i\]
\[ 145834 * 8 + 340216 * 4 + 145834 * 4 = 18863272\] Weight of dense
matrix \[ w * d * s_n\] \[ (340216 * 243) * 8 = 661379904\]

\begin{Shaded}
\begin{Highlighting}[]
\NormalTok{n =}\StringTok{ }\KeywordTok{nrow}\NormalTok{(weights)}
\NormalTok{w =}\StringTok{ }\KeywordTok{length}\NormalTok{(}\KeywordTok{unique}\NormalTok{(weights}\OperatorTok{$}\NormalTok{word_index))}
\NormalTok{d =}\StringTok{ }\KeywordTok{length}\NormalTok{(}\KeywordTok{unique}\NormalTok{(weights}\OperatorTok{$}\NormalTok{agency_index))}
\NormalTok{si =}\StringTok{ }\DecValTok{4}
\NormalTok{sd =}\StringTok{ }\DecValTok{8}

\CommentTok{# Triple rep}
\KeywordTok{cat}\NormalTok{(}\KeywordTok{paste}\NormalTok{(}\StringTok{"Triple Rep Est:"}\NormalTok{, n }\OperatorTok{*}\StringTok{ }\NormalTok{(si }\OperatorTok{+}\StringTok{ }\NormalTok{si }\OperatorTok{+}\StringTok{ }\NormalTok{sd), }\StringTok{"bytes }\CharTok{\textbackslash{}n}\StringTok{"}\NormalTok{))}
\end{Highlighting}
\end{Shaded}

\begin{verbatim}
## Triple Rep Est: 23336544 bytes
\end{verbatim}

\begin{Shaded}
\begin{Highlighting}[]
\KeywordTok{cat}\NormalTok{(}\StringTok{"Triple Rep Obs: "}\NormalTok{); }\KeywordTok{object.size}\NormalTok{(weights)}
\end{Highlighting}
\end{Shaded}

\begin{verbatim}
## Triple Rep Obs:
\end{verbatim}

\begin{verbatim}
## 23337544 bytes
\end{verbatim}

\begin{Shaded}
\begin{Highlighting}[]
\CommentTok{# Sparse}
\NormalTok{sm =}\StringTok{ }\KeywordTok{sparseMatrix}\NormalTok{(}\DataTypeTok{j =}\NormalTok{ weights}\OperatorTok{$}\NormalTok{agency_index, }\DataTypeTok{i =}\NormalTok{ weights}\OperatorTok{$}\NormalTok{word_index, }\DataTypeTok{x =}\NormalTok{ weights}\OperatorTok{$}\NormalTok{weight)}
\KeywordTok{cat}\NormalTok{(}\KeywordTok{paste}\NormalTok{(}\StringTok{"Sparse Est:"}\NormalTok{,  n }\OperatorTok{*}\StringTok{ }\NormalTok{sd }\OperatorTok{+}\StringTok{ }\NormalTok{n }\OperatorTok{*}\StringTok{ }\NormalTok{si }\OperatorTok{+}\StringTok{ }\NormalTok{d }\OperatorTok{*}\StringTok{ }\NormalTok{si, }\StringTok{"bytes }\CharTok{\textbackslash{}n}\StringTok{"}\NormalTok{))}
\end{Highlighting}
\end{Shaded}

\begin{verbatim}
## Sparse Est: 17503380 bytes
\end{verbatim}

\begin{Shaded}
\begin{Highlighting}[]
\KeywordTok{cat}\NormalTok{(}\StringTok{"Sparse Obs: "}\NormalTok{); }\KeywordTok{object.size}\NormalTok{(sm)}
\end{Highlighting}
\end{Shaded}

\begin{verbatim}
## Sparse Obs:
\end{verbatim}

\begin{verbatim}
## 17504880 bytes
\end{verbatim}

\begin{Shaded}
\begin{Highlighting}[]
\CommentTok{# Sparse transpose}
\KeywordTok{cat}\NormalTok{(}\KeywordTok{paste}\NormalTok{(}\StringTok{"Sparse Transpose Est:"}\NormalTok{, n }\OperatorTok{*}\StringTok{ }\NormalTok{sd }\OperatorTok{+}\StringTok{ }\NormalTok{w }\OperatorTok{*}\StringTok{ }\NormalTok{si }\OperatorTok{+}\StringTok{ }\NormalTok{n }\OperatorTok{*}\StringTok{ }\NormalTok{si, }\StringTok{"bytes }\CharTok{\textbackslash{}n}\StringTok{"}\NormalTok{))}
\end{Highlighting}
\end{Shaded}

\begin{verbatim}
## Sparse Transpose Est: 18863272 bytes
\end{verbatim}

\begin{Shaded}
\begin{Highlighting}[]
\KeywordTok{cat}\NormalTok{(}\StringTok{"Sparse Transpose Obs: "}\NormalTok{); }\KeywordTok{object.size}\NormalTok{(}\KeywordTok{t}\NormalTok{(sm))}
\end{Highlighting}
\end{Shaded}

\begin{verbatim}
## Sparse Transpose Obs:
\end{verbatim}

\begin{verbatim}
## 18864776 bytes
\end{verbatim}

\begin{Shaded}
\begin{Highlighting}[]
\CommentTok{# Dense}
\KeywordTok{cat}\NormalTok{(}\KeywordTok{paste}\NormalTok{(}\StringTok{"Dense Est:"}\NormalTok{, (w }\OperatorTok{*}\StringTok{ }\NormalTok{d) }\OperatorTok{*}\StringTok{ }\NormalTok{sd, }\StringTok{"bytes }\CharTok{\textbackslash{}n}\StringTok{"}\NormalTok{))}
\end{Highlighting}
\end{Shaded}

\begin{verbatim}
## Dense Est: 661379904 bytes
\end{verbatim}

\begin{Shaded}
\begin{Highlighting}[]
\KeywordTok{cat}\NormalTok{(}\StringTok{"Dense Obs: "}\NormalTok{); }\KeywordTok{object.size}\NormalTok{(}\KeywordTok{as}\NormalTok{(sm, }\StringTok{"dgeMatrix"}\NormalTok{))}
\end{Highlighting}
\end{Shaded}

\begin{verbatim}
## Dense Obs:
\end{verbatim}

\begin{verbatim}
## 661381080 bytes
\end{verbatim}

\textbf{3. Comment on the sizes of the objects you calculated and
verified above. Here are some questions to get you thinking:}

\begin{itemize}
\tightlist
\item
  How do the sizes compare to the sparse representation on disk in ASCII
  text, the weights.csv file?
\end{itemize}

\begin{Shaded}
\begin{Highlighting}[]
\KeywordTok{cat}\NormalTok{(}\KeywordTok{paste}\NormalTok{(}\StringTok{"File Size on Disk:"}\NormalTok{, }\KeywordTok{file.info}\NormalTok{(}\StringTok{"weights.csv"}\NormalTok{)}\OperatorTok{$}\NormalTok{size, }\StringTok{"bytes"}\NormalTok{))}
\end{Highlighting}
\end{Shaded}

\begin{verbatim}
## File Size on Disk: 44330812 bytes
\end{verbatim}

\begin{itemize}
\tightlist
\item
  What is the sparsity of the matrix? (Sparsity is the number of
  zero-valued elements divided by the total number of elements)
\end{itemize}

\begin{Shaded}
\begin{Highlighting}[]
\KeywordTok{cat}\NormalTok{(}\KeywordTok{paste}\NormalTok{(}\StringTok{"Sparsity:"}\NormalTok{, }\KeywordTok{round}\NormalTok{(((w }\OperatorTok{*}\StringTok{ }\NormalTok{d) }\OperatorTok{-}\StringTok{ }\NormalTok{n) }\OperatorTok{/}\StringTok{ }\NormalTok{(w }\OperatorTok{*}\StringTok{ }\NormalTok{d), }\DecValTok{3}\NormalTok{)))}
\end{Highlighting}
\end{Shaded}

\begin{verbatim}
## Sparsity: 0.982
\end{verbatim}

\begin{itemize}
\tightlist
\item
  What's the most efficient memory representation for this particular
  data?
\end{itemize}

The sparse matrix is the most efficient memory representation for this
particular dataset. In particular, the sparse matrix in row oriented
format appears to be the most efficient storage method.

\begin{itemize}
\tightlist
\item
  Under what conditions would a different representation work better?
  Could dense ever be better than sparse?
\end{itemize}

Dense could be better than sparse as the number of non-zero cells in the
matrix approaches zero. In the row-organized sparse representation of
the matrices in this example, each matrix object contains a vector of
non-zero entries, a vector of the row positions of these non-zero
entries, and a vector of column indices. In a case where a majority of
cells in the matrix are nonzero, the size of a matrix object with this
data storage format will be far greater than a dense matrix, which only
needs a vector of entries (i * j in length) and a 2-d vector specifying
its dimensions.

Hints: The Matrix package uses a variant of compressed sparse row matrix
representation, which you can read about. Inspect the matrices using
str.

\hypertarget{clustering}{%
\section{2. Clustering}\label{clustering}}

Let X be the weights matrix with columns for each agency and rows for
each word. I normalized X such that columns have L2 norm equal to 1.
This lets us compute a measure of similarity or correlation between
agencies by taking the dot product of the columns representing those
agencies. We can compute all the pairwise similarities simultaneously
with XT X. Values range between 0 and 1; a value of 0 means the agencies
share no words at all, and a value of 1 means the agencies give exactly
the same weight to each word. Thus 2D = 1 − XT X acts like a distance
matrix between the agencies, where 1 is a matrix with every entry equal
to scalar 1.

\textbf{1. Is crossprod faster than explicitly computing XT X? Why?}

\begin{Shaded}
\begin{Highlighting}[]
\KeywordTok{cat}\NormalTok{(}\StringTok{"XT X system time:}\CharTok{\textbackslash{}n}\StringTok{"}\NormalTok{)}
\end{Highlighting}
\end{Shaded}

\begin{verbatim}
## XT X system time:
\end{verbatim}

\begin{Shaded}
\begin{Highlighting}[]
\KeywordTok{system.time}\NormalTok{(}\KeywordTok{t}\NormalTok{(sm) }\OperatorTok\StringTok{ }\NormalTok{sm)}
\end{Highlighting}
\end{Shaded}

\begin{verbatim}
##    user  system elapsed 
##    0.45    0.02    0.46
\end{verbatim}

\begin{Shaded}
\begin{Highlighting}[]
\KeywordTok{cat}\NormalTok{(}\StringTok{"Crossprod system time:}\CharTok{\textbackslash{}n}\StringTok{"}\NormalTok{)}
\end{Highlighting}
\end{Shaded}

\begin{verbatim}
## Crossprod system time:
\end{verbatim}

\begin{Shaded}
\begin{Highlighting}[]
\KeywordTok{system.time}\NormalTok{(}\KeywordTok{crossprod}\NormalTok{(sm, sm))}
\end{Highlighting}
\end{Shaded}

\begin{verbatim}
##    user  system elapsed 
##    0.51    0.00    0.54
\end{verbatim}

From crossprod()'s manual:

The functions crossprod and tcrossprod are matrix products or ``cross
products'', ideally implemented efficiently without computing t(.)'s
unnecessarily.

In this case, the time is roughly the same, likely due to the fact that
the computation of t(.) is quick given its sparsity

\textbf{2. Is crossprod faster on the sparse version of X compared to
the dense version of X? Why?}

\begin{Shaded}
\begin{Highlighting}[]
\KeywordTok{cat}\NormalTok{(}\StringTok{"Sparse Time: }\CharTok{\textbackslash{}n}\StringTok{"}\NormalTok{)}
\end{Highlighting}
\end{Shaded}

\begin{verbatim}
## Sparse Time:
\end{verbatim}

\begin{Shaded}
\begin{Highlighting}[]
\KeywordTok{system.time}\NormalTok{(}\KeywordTok{crossprod}\NormalTok{(sm, sm))}
\end{Highlighting}
\end{Shaded}

\begin{verbatim}
##    user  system elapsed 
##    0.44    0.01    0.46
\end{verbatim}

\begin{Shaded}
\begin{Highlighting}[]
\KeywordTok{cat}\NormalTok{(}\StringTok{"Dense Matrix Time }\CharTok{\textbackslash{}n}\StringTok{"}\NormalTok{)}
\end{Highlighting}
\end{Shaded}

\begin{verbatim}
## Dense Matrix Time
\end{verbatim}

\begin{Shaded}
\begin{Highlighting}[]
\KeywordTok{system.time}\NormalTok{(}\KeywordTok{crossprod}\NormalTok{(}\KeywordTok{as}\NormalTok{(sm, }\StringTok{"dgeMatrix"}\NormalTok{), }\KeywordTok{as}\NormalTok{(sm, }\StringTok{"dgeMatrix"}\NormalTok{)))}
\end{Highlighting}
\end{Shaded}

\begin{verbatim}
##    user  system elapsed 
##   41.13    1.28   43.76
\end{verbatim}

\textbf{3. What is the range of similarity scores that appear between
different agencies? What two agencies are most similar? What does this
mean in terms of the words they are using?}

\begin{Shaded}
\begin{Highlighting}[]
\NormalTok{simmat =}\StringTok{ }\KeywordTok{crossprod}\NormalTok{(sm, sm)}
\NormalTok{simmat}\OperatorTok{@}\NormalTok{x =}\StringTok{ }\DecValTok{1} \OperatorTok{-}\StringTok{ }\NormalTok{simmat}\OperatorTok{@}\NormalTok{x}
\KeywordTok{hist}\NormalTok{(}\KeywordTok{as.vector}\NormalTok{(simmat[}\KeywordTok{upper.tri}\NormalTok{(simmat, }\DataTypeTok{diag =} \OtherTok{FALSE}\NormalTok{)]),}
     \DataTypeTok{main =} \StringTok{"Distribution of Distance Values"}\NormalTok{)}
\end{Highlighting}
\end{Shaded}

\begin{verbatim}
## <sparse>[ <logic> ] : .M.sub.i.logical() maybe inefficient
\end{verbatim}

\includegraphics{HW3_Final_files/figure-latex/unnamed-chunk-7-1.pdf}

\begin{Shaded}
\begin{Highlighting}[]
\NormalTok{simdf =}\StringTok{ }\KeywordTok{summary}\NormalTok{(simmat)}
\NormalTok{simdf =}\StringTok{ }\NormalTok{simdf[simdf}\OperatorTok{$}\NormalTok{i }\OperatorTok{!=}\StringTok{ }\NormalTok{simdf}\OperatorTok{$}\NormalTok{j,]}
\CommentTok{# head(simdf[order(simdf$x),])}

\KeywordTok{cat}\NormalTok{(}\StringTok{"2 Most Similar Agencies"}\NormalTok{)}
\end{Highlighting}
\end{Shaded}

\begin{verbatim}
## 2 Most Similar Agencies
\end{verbatim}

\begin{Shaded}
\begin{Highlighting}[]
\NormalTok{agencies}\OperatorTok{$}\NormalTok{agency_name[}\DecValTok{163}\NormalTok{]}
\end{Highlighting}
\end{Shaded}

\begin{verbatim}
## [1] "Office of Job Corps"
\end{verbatim}

\begin{Shaded}
\begin{Highlighting}[]
\NormalTok{agencies}\OperatorTok{$}\NormalTok{agency_name[}\DecValTok{122}\NormalTok{]}
\end{Highlighting}
\end{Shaded}

\begin{verbatim}
## [1] "Employment and Training Administration"
\end{verbatim}

\textbf{4. Fit an agglomerative clustering model using
cluster::agnes(as.dist(D)). Agglomerative clustering iteratively builds
clusters by adding points to groups. What 2 agencies are grouped
together first? Is this what you expected based on the previous
question?}

First negative terms = first 2 First positive term = index of pair First
2 Terms = 163, 122

\begin{Shaded}
\begin{Highlighting}[]
\CommentTok{# Appears that agnes needs the distance matrix to be 1 - dist to function properly}
\NormalTok{dmat =}\StringTok{ }\KeywordTok{as.dist}\NormalTok{(simmat)}
\NormalTok{clust =}\StringTok{ }\NormalTok{cluster}\OperatorTok{::}\KeywordTok{agnes}\NormalTok{(dmat, }\DataTypeTok{method =} \StringTok{'complete'}\NormalTok{)}

\NormalTok{clust}\OperatorTok{$}\NormalTok{merge[}\DecValTok{1}\OperatorTok{:}\DecValTok{10}\NormalTok{,]}
\end{Highlighting}
\end{Shaded}

\begin{verbatim}
##       [,1] [,2]
##  [1,] -122 -163
##  [2,] -148 -226
##  [3,]   -4 -212
##  [4,]    2 -180
##  [5,]  -30  -65
##  [6,]  -92 -139
##  [7,]  -91 -202
##  [8,] -155 -164
##  [9,]    4 -211
## [10,]   -6 -151
\end{verbatim}

\begin{Shaded}
\begin{Highlighting}[]
\NormalTok{agencies[}\DecValTok{122}\NormalTok{,]}
\end{Highlighting}
\end{Shaded}

\begin{verbatim}
##                                agency_name agency_id      total
## 122 Employment and Training Administration       291 8822747603
\end{verbatim}

\begin{Shaded}
\begin{Highlighting}[]
\NormalTok{agencies[}\DecValTok{163}\NormalTok{,]}
\end{Highlighting}
\end{Shaded}

\begin{verbatim}
##             agency_name agency_id       total
## 163 Office of Job Corps       292 16194769391
\end{verbatim}

\textbf{5. What is the first group of 3 agencies? The first group of 4
agencies? Does agglomerative clustering appear to be doing something
reasonable?}

First 3 Terms = 148, 226, 180 First 4 Terms = 148, 226, 180, 211

\begin{Shaded}
\begin{Highlighting}[]
\NormalTok{agencies[}\KeywordTok{c}\NormalTok{(}\DecValTok{138}\NormalTok{, }\DecValTok{226}\NormalTok{, }\DecValTok{180}\NormalTok{, }\DecValTok{211}\NormalTok{),]}
\end{Highlighting}
\end{Shaded}

\begin{verbatim}
##                                agency_name agency_id      total
## 138           Foreign Agricultural Service       154 1985800923
## 226            Office of Inspector General       459  118954118
## 180            Office of Inspector General       270  144048587
## 211 Assistant Secretary for Administration       886   65014325
\end{verbatim}

\begin{Shaded}
\begin{Highlighting}[]
\NormalTok{factoextra}\OperatorTok{::}\KeywordTok{fviz_dend}\NormalTok{(clust)}
\end{Highlighting}
\end{Shaded}

\includegraphics{HW3_Final_files/figure-latex/unnamed-chunk-9-1.pdf}

\textbf{6. Fit a partitioning around medoids clustering model using
cluster::pam with k = 2 clusters. To what extent do the cluster
assignments agree with the agglomerative clustering model?}

\begin{Shaded}
\begin{Highlighting}[]
\NormalTok{hlabels <-}\StringTok{ }\KeywordTok{cutree}\NormalTok{(clust, }\DecValTok{2}\NormalTok{)}
\NormalTok{pamclust <-}\StringTok{ }\NormalTok{cluster}\OperatorTok{::}\KeywordTok{pam}\NormalTok{(}\DataTypeTok{k =} \DecValTok{2}\NormalTok{, dmat)}
\NormalTok{pamlabels <-}\StringTok{ }\NormalTok{pamclust}\OperatorTok{$}\NormalTok{clustering}

\NormalTok{dat =}\StringTok{ }\KeywordTok{cbind}\NormalTok{(agencies, pamlabels, hlabels)}

\NormalTok{dat }\OperatorTok\StringTok{ }\KeywordTok{ggplot}\NormalTok{(}\KeywordTok{aes}\NormalTok{(}\DataTypeTok{x =} \KeywordTok{factor}\NormalTok{(pamlabels),}
                   \DataTypeTok{fill =} \KeywordTok{factor}\NormalTok{(hlabels))) }\OperatorTok{+}\StringTok{ }
\StringTok{  }\KeywordTok{geom_bar}\NormalTok{() }\OperatorTok{+}
\StringTok{  }\KeywordTok{xlab}\NormalTok{(}\StringTok{"PAM Cluster Label"}\NormalTok{) }\OperatorTok{+}
\StringTok{  }\KeywordTok{ylab}\NormalTok{(}\StringTok{"Count"}\NormalTok{) }\OperatorTok{+}
\StringTok{  }\KeywordTok{guides}\NormalTok{(}\DataTypeTok{fill=}\KeywordTok{guide_legend}\NormalTok{(}\DataTypeTok{title=}\StringTok{"Agnes Label"}\NormalTok{)) }\OperatorTok{+}
\StringTok{  }\KeywordTok{ggtitle}\NormalTok{(}\StringTok{"Cluter Method Comparison"}\NormalTok{)}
\end{Highlighting}
\end{Shaded}

\includegraphics{HW3_Final_files/figure-latex/unnamed-chunk-10-1.pdf}

\begin{Shaded}
\begin{Highlighting}[]
\NormalTok{viz <-}\StringTok{ }\NormalTok{MASS}\OperatorTok{::}\KeywordTok{isoMDS}\NormalTok{(dmat, }\DataTypeTok{k=}\DecValTok{2}\NormalTok{) }\CommentTok{# k is the number of dim}
\end{Highlighting}
\end{Shaded}

\begin{verbatim}
## initial  value 43.497777 
## iter   5 value 33.605293
## iter  10 value 30.562769
## iter  15 value 29.739395
## iter  20 value 29.495166
## final  value 29.430899 
## converged
\end{verbatim}

\begin{Shaded}
\begin{Highlighting}[]
\KeywordTok{plot}\NormalTok{(viz}\OperatorTok{$}\NormalTok{points[,}\DecValTok{1}\NormalTok{], }
\NormalTok{     viz}\OperatorTok{$}\NormalTok{points[,}\DecValTok{2}\NormalTok{], }
     \DataTypeTok{xlab=}\StringTok{"Coordinate 1"}\NormalTok{, }\DataTypeTok{ylab=}\StringTok{"Coordinate 2"}\NormalTok{, }
     \DataTypeTok{main=}\StringTok{"NonMetric MDS"}\NormalTok{,}
     \DataTypeTok{col =}\NormalTok{ pamlabels)}
\end{Highlighting}
\end{Shaded}

\includegraphics{HW3_Final_files/figure-latex/unnamed-chunk-10-2.pdf}

\textbf{7. Think about all the steps we've taken in preparing the data
and coming this far. Would you say that clustering is a subjective
task?}

Yes


\end{document}
